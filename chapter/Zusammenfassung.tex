\chapter{Zusammenfassung}
\label{ch:Zusammenfassung}

ASAs sind nur ein Angriffsvektor gegen symmetrische Verschlüssel"-ungen, doch liegen durchaus im Bereich des Möglichen. Wie in Kapitel \ref{ch:Einleitung} erwähnt, finden wahrscheinlich derartige Angriffe statt. Jederzeit könnten symmetrische Verschlüsselungen im Internet abgehört werden, ohne dass jemand etwas bemerkt. Schemen mit öffentlich übertragenen nonce sind generell gefährdet von ASAs, aber auch Schemen mit Zufallskomponenten, bei denen kein öffentlicher nonce übertragen wird sind durch die allgemeinere biased-ciphertext attack verwundbar. In dem Paper konnte gezeigt werden, dass die Klasse der unique ciphertext schemes eine sinnvolle Gegenmaßnahme gegen ASAs bilden. Einzig und allein das Scheitern beim Entschlüsseln enthüllt bei diesen Schemen Subversionen. 

Wer Interesse an einen praktischen Ansatz hat, kann sich mit einem weiteren Paper der Stanford University beschäftigen (\cite{GBPG}). The Design and Implementation of Protocol-Based Hidden Key Recovery behandelt unter anderem key recovery in SSL/TLS. Es wird gezeigt, wie key recovery in bestehende Protokolle wie SSL/TLS und SSH integriert wird ohne die Protokolle selbst zu verändern. Eine andere Arbeit von Mihir Bellare, Joseph Jaeger und Daniel Kane, Mass-surveillance without the State: Strongly Undetectable Algorithm-Substitution Attacks (\cite{BJK}), beschäftigt sich weiterführend mit der biased ciphertext attack. Sie soll zeigen, dass ASAs noch viel mächtiger sein können als sie ohnehin schon sind. Als Einstieg in das Thema und einen kurzen Überblick bietet Kenneth G. Paterson in einer Präsentation auf der Crypto 2014 (\cite{praesi}). In knapp 20 Minuten gibt er eine Zusammenfassung des Originalwerks.