\chapter*{Kurzzusammenfassung}

\begin{centering}
\textbf { \LARGE
Security of Symmetric Encryption against Mass Surveillance
\\[1.2cm]
}
\end{centering}

{\large
Zusammenfassung
}

{
Spätestens seit den Snowden Leaks ist öffentlich bekannt, dass Massen"-über"-wachung im Internet stattfindet. Codenamen wie PRISM werden in Zeitschriften ver"-öffentlicht (\cite{Guard}). Wir sehen, dass verschlüsselte Übertragungen unsere Daten nicht vor unbefugtem Zugriff schützen. Pseudozufallsgeneratoren die möglicherweise nicht ganz zufällig sind, wie zum Beispiel der NIST Dual EC DRBG, sorgen für potenziell unsichere VPN Verbindungen und Backdoors in eigentlich sicheren Übertragungen (\cite{dual}) könnten in nahezu jeder Anwendung vorkommen. Das original Paper von Mihir Bellare, Kenneth G. Paterson und Phillip Rogaway soll die erste Salve im Kampf gegen Massenüberwachung abgeben (\cite{praesi}) und fokussiert sich auf symmetrische Verschlüsselungsschemen und einen speziellen Angriffsvektor, den sogenannte Algorithm Substitution Attacks. Es soll eine Grundlage mit der Definition und Formalisierung schaffen für die Abwehr dieses Angriffs. Denn symmetrische Verschlüsselungen bilden das Rückkrad vieler alltäglicher Übertragungs"-techniken wie zum Beispiel IPsec oder TLS. Diese Arbeit soll zunächst einen Überblick über das Originalwerk geben, Inhalte zusammenfassen und verdeutlichen. Es wird dargestellt, wie Algorithm Substitution Attacks funktionieren könnten und welchen symmetrische Schemen dagegen gefeit sind.
}\vspace{12pt}

{\large
Stichworte
}

{
Algorithm Substitution Attacks, symmetrische Verschlüsselung, unique ciphertext scheme, key recovery
}