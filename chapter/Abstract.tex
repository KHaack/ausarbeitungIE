\begin{centering}
\textbf { \LARGE
Hypertext-Systeme 1
\\(Lokale Systeme)
\\[1.2cm]
}
\end{centering}

{\large
Abstract
}

{
In den 1980er Jahren entwickelten sich lokale (nicht zwangsweise netzbasierte) Hypertext-Systeme, wie Intermedia, HyperTIES und Hypercard.

hypertext is
an approach to information management in which data
is stored in a network of nodes connected by links.
Nodes can contain text, graphics, audio, video, as well

Während der Begriff \glqq Hypertext\grqq{ }von Ted Nelson in den 1960ern geprägt wurde \cite{Nelson1965}, kann das Konzept von Hypertext auf die Beschreibung der \glqq Memex\grqq{ }von Vannevar Bush aus dem Jahr 1945 zurückgefühert werden \cite{Bush1945}.

\begin{quote}
	\glqq Consider a future device for individual use, which is a sort of mechanized private file and library. [...] A memex is a device in which an individual stores all his books, records, and communications, and which is mechanized so that it may be consulted with exceeding speed and flexibility. It is an enlarged intimate supplement to his memory.\grqq{ }\cite{Bush1945}
\end{quote}

Fall Sie dieses Dokument in lesen, verwenden Sie auch Hypertext
Hypertext kann man sich kaum vorstellen ohne Internet
	wie könnte das sonst aussehen?

Diese Arbeit beschäftigt sich mit der Nutzungsgeschichte und ihre Entwicklung über die Zeit.
Wie technische Entwicklungen Nutzungsformen hat entstehen lassen. 
Wie Probleme wie \glqq dandling edges\grqq{ } oder \glqq lost in hyperspace\grqq{ } entstehen und ob oder wie die Systeme mit diesen Problemen umgehen.

}
