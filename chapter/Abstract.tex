\begin{centering}
\textbf { \LARGE
Hypertext-Systeme 1
\\(Lokale Systeme)
\\[1.2cm]
}
\end{centering}

{\large
Abstract
}

{

\begin{quote}
	\glqq Consider a future device for individual use, which is a sort of mechanized private file and library. [...] A memex is a device in which an individual stores all his books, records, and communications, and which is mechanized so that it may be consulted with exceeding speed and flexibility. It is an enlarged intimate supplement to his memory.\grqq{ } aus Vannevar Bushs Arbeit \glqq As we may think\grqq{ }, 1945\cite{Bush1945}
\end{quote}

Während der Begriff \glqq Hypertext\grqq{ }erst in 1960er Jahren von Theodor Holm Nelson \cite{Nelson1965} geprägt wurde, er dachte Vannevar Bush in seiner Arbeit \glqq As we may think\grqq{ }die Memex. Das Konzept der Memex beschreibt eben genau diese Hypertext-Funktionalitäten, bevor die technischen Möglichkeiten überhaupt an so etwas denken ließen. Selbst jetzt in diesem Moment verwenden Sie als Leser ein Hypertext-Dokument. Je nachdem ob dieses Dokument ausgedruckt ist oder eine Anwendung Ihnen das Dokument anzeigt, haben Sie verschiedene Möglichkeiten den Text zu lesen. In ausgedruckter Form dieses Dokumentes sehen sie eine Referenz wie am Ende des Zitats und können ins Literaturverzeichnis in die entsprechende Zeile blättern. Von dort aus können Sie selbst entscheiden, ob Sie in der Quelle weiter lesen. In digitaler Form reicht meistens ein Klick und die Anwendung springt für Sie auf die entsprechende Seite. Doch diese Arbeit beschäftigt sich nicht mit den heutigen Möglichkeiten, sondern eher wie das Konzept Hypertext angefangen hat, was im Laufe der Zeit daraus geworden ist und welche Probleme dieses Konzept mit sich bringt.

}
